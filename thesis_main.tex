%-----------------------------------
% Define document and include general packages
%-----------------------------------
\documentclass[12pt,oneside,titlepage,listof=totoc,bibliography=totoc]{scrartcl}
\usepackage[utf8]{inputenc}
\usepackage[ngerman]{babel}
\usepackage[babel,german=quotes]{csquotes}
\usepackage[T1]{fontenc}
\usepackage{fancyhdr}
\usepackage{fancybox}
\usepackage[a4paper, left=4cm, right=2cm, top=2.8cm, bottom=2.3cm]{geometry}
\usepackage{graphicx}
\usepackage{colortbl}
\usepackage{array}
\usepackage{float}      %Positionierung von Abb. und Tabellen mit [H] erzwingen
\usepackage{footnote}
\usepackage[singlelinecheck=false, justification=RaggedRight, labelfont=bf]{caption}
\usepackage{enumitem}
\usepackage{amssymb}
\usepackage{mathptmx}
% Behebt, zusammen mit Package courier, pixelige Überschriften. Ist, zusammen mit mathptx, dem times-Package vorzuziehen. Details: https://latex-kurs.de/fragen/schriftarten/Times_New_Roman.html
\usepackage{helvet}\renewcommand{\familydefault}{\sfdefault}\fontfamily{phv}\selectfont   % Custom: Helvetica
\usepackage{courier}
\usepackage{amsmath}
\usepackage[table]{xcolor}
\usepackage{marvosym}			% Verwendung von Symbolen, z.B. perfektes Eurozeichen
\usepackage[colorlinks=true,linkcolor=black]{hyperref}
\definecolor{darkblack}{rgb}{0,0,0}
\hypersetup{colorlinks=true, breaklinks=true, linkcolor=darkblack, menucolor=darkblack, urlcolor=darkblack}

% original:
% \fontfamily{ptm}\selectfont

\usepackage{ragged2e}

% Mehrere Fussnoten nacheinander mit Komma separiert
\usepackage[hang, multiple]{footmisc}
\setlength{\footnotemargin}{1em}

% todo Aufgaben als Kommentare verfassen für verschiedene Editoren
\usepackage{todonotes}
\reversemarginpar % damit Todos links im freien Raum angezeigt werden und nicht rechts

%Pakete für Tabellen
\usepackage{epstopdf}
\usepackage{nicefrac} % Brüche
\usepackage{multirow}
\usepackage{rotating} % vertikal schreiben
\usepackage{mdwlist}

\definecolor{dunkelgrau}{rgb}{0.8,0.8,0.8}
\definecolor{hellgrau}{rgb}{0.0,0.7,0.99}
% Colors for listings
\definecolor{mauve}{rgb}{0.58,0,0.82}
\definecolor{dkgreen}{rgb}{0,0.6,0}

% sauber formatierter Quelltext
\usepackage{listings}
\lstset{
	captionpos=t,
	extendedchars=true,
	numbers=left,
	numberstyle=\tiny,
	numbersep=5pt,
	breaklines=true,
	showstringspaces=false,
	frame=l ,
	xleftmargin=5pt,
	xrightmargin=5pt,
	basicstyle=\ttfamily\scriptsize,
	stepnumber=1,
	keywordstyle=\color{blue},          % keyword style
  	commentstyle=\color{dkgreen},       % comment style
  	stringstyle=\color{mauve}         % string literal style
}

%-----------------------------------------------
% Quelltextverzeichnis und Referenzen darauf
%-----------------------------------------------
\renewcommand\lstlistingname{Quelltext} % changes the caption name for listings.
\renewcommand\lstlistlistingname{Quelltextverzeichnis} % changes the header name for the list of listings
\def\lstlistingautorefname{Quelltext} % changes the autoref reference text

% Einigermaßen sauber formatierter JavaScript und TypeScript Quelltext
%-----------------------------------------------
% JavaScript als Programmiersprache unterstützen
%-----------------------------------------------
\definecolor{lightgray}{rgb}{.9,.9,.9}
\definecolor{darkgray}{rgb}{.4,.4,.4}
\definecolor{purple}{rgb}{0.65, 0.12, 0.82}
\lstdefinelanguage{JavaScript}{
  keywords={break, case, catch, class, const, continue, debugger, default, delete, do, else, extends, false, finally, for, function, if, in, instanceof, new, null, return, super, switch, this, true, try, typeof, var, void, while, with, yield, let, static, await, },
  morecomment=[l]{//},
  morecomment=[s]{/*}{*/},
  morestring=[b]',
  morestring=[b]",
  ndkeywords={class, enum, export, boolean, throw, implements, import, this, interface, public, private, protected, package, static, abstract},
  keywordstyle=\color{blue}\bfseries,
  ndkeywordstyle=\color{darkgray}\bfseries,
  identifierstyle=\color{black},
  commentstyle=\color{purple}\ttfamily,
  stringstyle=\color{red}\ttfamily,
  sensitive=true
}

%-----------------------------------------------
% TypeScript als Programmiersprache unterstützen
%-----------------------------------------------
\lstdefinelanguage{TypeScript}{
  keywords={as, any, of, from, type, symbol, async, break, case, catch, class, const, continue, debugger, default, delete, do, else, extends, false, finally, for, function, if, in, instanceof, new, null, return, super, switch, this, true, try, typeof, var, while, with, yield, let, static, await },
  morecomment=[l]{//},
  morecomment=[s]{/*}{*/},
  morestring=[b]',
  morestring=[b]",
  ndkeywords={class, constructor, enum, export, declare, module, namespace, boolean, string, number, void, throw, implements, require, import, this, interface, public, private, protected, package, static, abstract, set, get},
  keywordstyle=\color{blue}\bfseries,
  ndkeywordstyle=\color{darkgray}\bfseries,
  identifierstyle=\color{black},
  commentstyle=\color{purple}\ttfamily,
  stringstyle=\color{red}\ttfamily,
  sensitive=true
}


% Biblatex
\usepackage[
backend=biber,
style=numeric,
citestyle=authoryear,
url=false,
isbn=false,
notetype=footonly,
hyperref=false,
sortlocale=de]{biblatex}

%weitere Anpassungen für BibLaTex
% Opptionen für Biblatex
\ExecuteBibliographyOptions{%
giveninits=false,
isbn=true,
url=true,
doi=false,
eprint=false,
maxbibnames=7, % Alle Autoren (kein et al.)
maxcitenames=2, % et al. ab dem 3. Autor
backref=false, % Rückverweise auf Zitatseiten
bibencoding=utf8, % wenn .bib in utf8, sonst ascii
bibwarn=true, % Warnung bei fehlerhafter bib-Datei
}%

% et al. an Stelle von u.a.
\DefineBibliographyStrings{ngerman}{
   andothers = {{et\,al\adddot}},
}

% Klammern um das Jahr in der Fußnote
\renewbibmacro*{cite:labelyear+extrayear}{%
  \iffieldundef{labelyear}
    {}
    {\printtext[bibhyperref]{%
       \mkbibparens{%
         \printfield{labelyear}%
         \printfield{extrayear}}}}}

\renewbibmacro*{cite:title}{%
  \printtext[bibhyperref]{%
    \printfield[citetitle]{labeltitle}%
    \setunit{\addcomma\space}%
    \printdate}}

\DeclareNameFormat{last-first}{%
  \iffirstinits
    {\usebibmacro{name:family-given}
        {\namepartfamily}
        {\namepartgiveni}
        {\namepartprefix}
        {\namepartsuffix}
    }
    {\usebibmacro{name:family-given}
        {\namepartfamily}
        {\namepartgiven}
        {\namepartprefix}
        {\namepartsuffix}
    }%
  \usebibmacro{name:andothers}}

% Alternative Notation der Fußnoten
% Zeigt sowohl den Nachnamen als auch den Vornamen an
% Beispiel: \fullfootcite[Vgl. ][Seite 5]{Tanenbaum.2003}
\DeclareCiteCommand{\fullfootcite}[\mkbibfootnote]
  {\usebibmacro{prenote}}
  {\usebibmacro{citeindex}%
    \printnames[sortname][1-1]{author}%
    \addspace (\printfield{year})}
  {\addsemicolon\space}
  {\usebibmacro{postnote}}

%Autoren (Nachname, Vorname)
\DeclareNameAlias{default}{family-given}

%Reihenfolge von publisher, year, address verändern
% Achtung, bisher nur für den Typ @book definiert

%% Definiert @Book Eintrag
\DeclareBibliographyDriver{book}{%
  \printnames{author}%
  \newunit\addcolon\space
  \printfield{title}%
  \setunit*{,\space}%
  \printfield{edition}%
  \setunit*{\addcomma\space}%
  \printlist{publisher}%
  \newunit\newblockpunct
  \printlist{location}%
  \setunit*{\space}%
  \printfield{year}%
  \setunit*{,\space}%
  \printfield{isbn}%
  \finentry}

%% Definiert @Online Eintrag
\DeclareBibliographyDriver{online}{%
  \printnames{author}%
  \newunit\newblockpunct
  \printfield{title}%
  \setunit*{,\space}%
  %\newunit\newblock
  \printfield{url}%
  \setunit*{,\space Erscheinungsjahr:\space}%
  \printfield{year}%
  \setunit*{,\space Aufruf am:\space}%
  \printfield{note}%
  \finentry}

%% Definiert @Article Eintrag
\DeclareBibliographyDriver{article}{%
  \printnames{author}%
  \newunit\newblockpunct
  \printfield{title}%
  \setunit*{.\space In:\space}%
  %\newunit\newblock
  \usebibmacro{journal}%
  \setunit*{\space (}%
  \printfield{year}\newunit{)}%
  \finentry}

%% Definiert @InProceedings Eintrag
\DeclareBibliographyDriver{inproceedings}{%
	\printnames{author}%
	\setunit*{,\space (}%
	\printfield{year}\newunit{)}%
	\newunit\newblockpunct
	\printfield{title}%
	\setunit*{\space}%
	\usebibmacro{booktitle}%
	\setunit*{,\space}%
	\printfield{isbn}%
	\setunit*{,\space}%
	\printfield{doi}%
	\finentry}

%Doppelpunkt nach dem letzten Autor
\renewcommand*{\labelnamepunct}{\addcolon\addspace }

%Komma an Stelle des Punktes
\renewcommand*{\newunitpunct}{\addcomma\space}

%Autoren durch Semikolon trennen
\newcommand*{\bibmultinamedelim}{\addsemicolon\space}%
\newcommand*{\bibfinalnamedelim}{\addsemicolon\space}%
\AtBeginBibliography{%
  \let\multinamedelim\bibmultinamedelim
  \let\finalnamedelim\bibfinalnamedelim
}

%Titel nicht kursiv anzeigen
\DeclareFieldFormat{title}{#1\isdot}



%Bib-Datei einbinden
\addbibresource{literatur/literatur.bib}

% Pfad fuer Abbildungen
\graphicspath{{./}{./abbildungen/}}

%-----------------------------------
% Weitere Ebene einfügen
\input{skripte/weitereEbene}

%-----------------------------------
% Zeilenabstand 1,5-zeilig
%-----------------------------------
\usepackage{setspace}
\onehalfspacing

%-----------------------------------
% Absätze durch eine neue Zeile
%-----------------------------------
\setlength{\parindent}{0mm}
\setlength{\parskip}{0.8em plus 0.5em minus 0.3em}

\sloppy					%Abstände variieren
\pagestyle{headings}

%-----------------------------------
% Abkürzungsverzeichnis
%-----------------------------------
\usepackage[intoc]{nomencl}
\renewcommand{\nomname}{Abkürzungsverzeichnis}
\setlength{\nomlabelwidth}{.20\textwidth}
\renewcommand{\nomlabel}[1]{#1 \dotfill}
\setlength{\nomitemsep}{-\parsep}
\makenomenclature

%-----------------------------------
% Meta informationen
%-----------------------------------
%-----------------------------------
% Meta Informationen zur Arbeit
%-----------------------------------

% Autor 
\newcommand{\myAutor}{Max Mustermann}

% Adresse
\newcommand{\myAdresse}{Heidestra\ss e 17 \\ \> \> \> 51147 Köln}

% Titel der Arbeit
\newcommand{\myTitel}{LATEX-Vorlage - mit Biblatex}

% Betreuer
\newcommand{\myBetreuer}{Prof. Dr. Peter Lustig}

% Lehrveranstaltung
\newcommand{\myLehrveranstaltung}{Modul Nr. 1}

% Matrikelnummer
\newcommand{\myMatrikelNr}{123456}

% Ort
\newcommand{\myOrt}{Düsseldorf}

% Datum der Abgabe
\newcommand{\myAbgabeDatum}{\today}

% Semesterzahl
\newcommand{\mySemesterZahl}{7}

% Name der Hochschule
\newcommand{\myHochschulName}{FOM Hochschule für Oekonomie \& Management Essen}

% Standort der Hochschule
\newcommand{\myHochschulStandort}{Düsseldorf}

% Studiengang
\newcommand{\myStudiengang}{Wirtschaftsinformatik}

% Art der Arbeit
\newcommand{\myThesisArt}{Bachelor Thesis}

% Zu erlangender akademische Grad
\newcommand{\myAkademischerGrad}{Bachelor of Science (B. Sc.)}

% Firma
\newcommand{\myFirma}{Mustermann GmbH}


\ifdefined\FOMEN
%Englisch
\entrue
\usepackage[english]{babel}
\else
%Deutsch
\detrue
\usepackage[ngerman]{babel}
\fi




%-----------------------------------
% PDF Meta Daten setzen
%-----------------------------------
\hypersetup{
    pdfinfo={
        Title={\myTitel},
        Subject={\myStudiengang},
        Author={\myAutor},
        Build=1.1
    }
}

%-----------------------------------
% Umlaute in Code korrekt darstellen
% siehe auch: https://en.wikibooks.org/wiki/LaTeX/Source_Code_Listings
%-----------------------------------
\lstset{literate=
	{á}{{\'a}}1 {é}{{\'e}}1 {í}{{\'i}}1 {ó}{{\'o}}1 {ú}{{\'u}}1
	{Á}{{\'A}}1 {É}{{\'E}}1 {Í}{{\'I}}1 {Ó}{{\'O}}1 {Ú}{{\'U}}1
	{à}{{\`a}}1 {è}{{\`e}}1 {ì}{{\`i}}1 {ò}{{\`o}}1 {ù}{{\`u}}1
	{À}{{\`A}}1 {È}{{\'E}}1 {Ì}{{\`I}}1 {Ò}{{\`O}}1 {Ù}{{\`U}}1
	{ä}{{\"a}}1 {ë}{{\"e}}1 {ï}{{\"i}}1 {ö}{{\"o}}1 {ü}{{\"u}}1
	{Ä}{{\"A}}1 {Ë}{{\"E}}1 {Ï}{{\"I}}1 {Ö}{{\"O}}1 {Ü}{{\"U}}1
	{â}{{\^a}}1 {ê}{{\^e}}1 {î}{{\^i}}1 {ô}{{\^o}}1 {û}{{\^u}}1
	{Â}{{\^A}}1 {Ê}{{\^E}}1 {Î}{{\^I}}1 {Ô}{{\^O}}1 {Û}{{\^U}}1
	{œ}{{\oe}}1 {Œ}{{\OE}}1 {æ}{{\ae}}1 {Æ}{{\AE}}1 {ß}{{\ss}}1
	{ű}{{\H{u}}}1 {Ű}{{\H{U}}}1 {ő}{{\H{o}}}1 {Ő}{{\H{O}}}1
	{ç}{{\c c}}1 {Ç}{{\c C}}1 {ø}{{\o}}1 {å}{{\r a}}1 {Å}{{\r A}}1
	{€}{{\EUR}}1 {£}{{\pounds}}1 {„}{{\glqq{}}}1
}

%-----------------------------------
% Kopfbereich / Header definieren
%-----------------------------------
\pagestyle{fancy}
\fancyhf{}
\fancyhead[C]{-\ \thepage\ -}						% Seitenzahl oben, mittig
%\fancyhead[L]{\leftmark}							% kein Footer vorhanden
\renewcommand{\headrulewidth}{0.4pt}


%-----------------------------------
% Start the document here:
%-----------------------------------
\begin{document}

\pagenumbering{Roman}								% Seitennummerierung auf römisch umstellen
\renewcommand{\refname}{Literaturverzeichnis}		% "Literatur" in
%"Literaturverzeichnis" umbenennen
\newcolumntype{C}{>{\centering\arraybackslash}X}	% Neuer Tabellen-Spalten-Typ:
%Zentriert und umbrechbar

%-----------------------------------
% Titlepage
%-----------------------------------
\begin{titlepage}
	\newgeometry{left=2cm, right=2cm, top=2cm, bottom=2cm}
	\begin{center}
		\textbf{\myHochschulName}\\
		\textbf{\myHochschulStandort}\\
		\vspace{1.5cm}
			\includegraphics[width=3cm]{abbildungen/fomLogo.jpg} \\
		\vspace{1.5cm}
		Berufsbegleitender Studiengang\\
		\myStudiengang, \mySemesterZahl. Semester\\
		\vspace{2cm}
		\textbf{\myThesisArt}\\
		\textbf{zur Erlangung des Grades eines}\\
		\textbf{\myAkademischerGrad}\\
		% Oder für Hausarbeiten:
		%\textbf{im Rahmen der Lehrveranstaltung}\\
		%\textbf{\myLehrveranstaltung}\\
		\vspace{2cm}
		über das Thema\\
		\Huge{\myTitel}\\
		\vspace{0.2cm}
	\end{center}
	\normalsize
	\vfill
	\begin{tabbing}
		Links \= Mitte \= Rechts\kill
		Betreuer: \> \> \myBetreuer\\
		\> \> \\

		Autor: \> \> \myAutor\\
		\> \>  Matrikelnr.: \myMatrikelNr\\
		\> \> \myAdresse\\
		\> \> \\
		Abgabe: \> \> \myAbgabeDatum
	\end{tabbing}
\end{titlepage}

%-------Ende Titelseite-------------

%-----------------------------------
% Sperrvermerk
%-----------------------------------
%\input{kapitel/anhang/sperrvermerk}

%-----------------------------------
% Inhaltsverzeichnis
%-----------------------------------
\setcounter{page}{2}
\tableofcontents
\newpage

%-----------------------------------
% Abkürzungsverzeichnis
%-----------------------------------
\printnomenclature
\newpage
%-----------------------------------
% Abbildungsverzeichnis
%-----------------------------------
\listoffigures
\newpage
%-----------------------------------
% Quellcodeverzeichnis
%-----------------------------------
\lstlistoflistings
\newpage
%-----------------------------------
% Tabellenverzeichnis
%-----------------------------------
\listoftables
\newpage
%-----------------------------------
% Seitennummerierung auf arabisch und ab 1 beginnend umstellen
%-----------------------------------
\pagenumbering{arabic}
\setcounter{page}{1}
%-----------------------------------
% Kapitel / Inhalte
%-----------------------------------
\section{Einleitung}
Dies soll eine \LaTeX{}-Vorlage für den persönlichen Gebrauch werden. Sie hat weder einen Anspruch auf Richtigkeit, noch auf Vollständigkeit. Die Quellen liegen auf Github zur allgemeinen Verwendung. Verbesserungen sind jederzeit willkommen.

\subsection{Zielsetzung}
Kleiner Reminder für mich in Bezug auf die Dinge, die wir bei der Thesis beachten sollten und \LaTeX{}-Vorlage für die Thesis.

\subsection{Aufbau der Arbeit}
Kapitel \ref{infos} enthält die Inhalte des Thesis-Days und alles, was zum inhaltlichen erstellen der Thesis relevant sein könnte. In Kapitel \ref{latexDetails} \nameref{latexDetails} findet ihr wichtige Anmerkungen zu \LaTeX{}, wobei die wirklich wichtigen Dinge im Quelltext dieses Dokumentes stehen (siehe auch die Verzeichnisstruktur in Abbildung \ref{fig:verzeichnisStruktur}).


% old
\begin{figure}[H]
  \begin{center}
    \includegraphics[width=0.9\textwidth]{verzeichnisStruktur}
    \caption{Verzeichnisstruktur der \LaTeX{}-Dateien}
  \end{center}
\end{figure}

% new from upstream
\begin{figure}[H]
 \centerline{\includegraphics[width=0.9\textwidth]{verzeichnisStruktur}}
 \caption{Verzeichnisstruktur der \LaTeX{}-Datein}\label{fig:verzeichnisStruktur}
 \floatfoot{Quelle: Eigene Darstellung}
\end{figure}

\newpage
\section{Informationen vom Thesis-Day} \label{infos}
Siehe auch Wissenschaftliches Arbeiten~\footcite[Vgl. ][S. 1]{Balzert.2008}. Damit sollten alle wichtigen Informationen abgedeckt sein ;-)

\subsection{Pre-Anmeldephase}
\subsubsection{Vorüberlegungen}
Trichtermethode: Man beginnt mit der eigentlichen  Konklusion und überlegt dann, welche allgemeinen Teile dafür benötigt werden.

Welchen Mehrwert soll die Arbeit bieten \footnote{Diese Fußnote hat inhaltlich keinen Sinn. Es soll nur ein langer Text generiert werden, dass dieser Vermerk über zwei Zeilen reicht und bündig dargestellt wird.}? Auch darüber nachdenken, wie die Arbeit einen selbst weiter bringen kann. Studienverlauf prüfen. Welche Vorlesungen hat mich besonders interessiert? Wo liegen meine Stärken etc.

\begin{enumerate}
\item Themenfindung
\item Literaturrecherche
\item Gliederung/Motivationspapier erstellen
\item Betreuerauswahl (siehe Liste im \ac{OC})
\item Anmeldung (ab 141 Credits möglich)
\end{enumerate}

\subsubsection{Anregungen finden}
\begin{itemize}
\item \href{http://www.diplom.de}{www.diplom.de}
\item \href{http://www.hausarbeiten.de}{www.hausarbeiten.de}
\item Datenbanken aus Tools and Methods
\item etc.
\end{itemize}

\newpage
\subsection{Anfertigungsphase}
Die Anmeldung ist mittlerweile jeden Mittwoch möglich.
\begin{figure}[H]
\begin{center}
\includegraphics[width=0.9\textwidth]{campusDownload}
\caption{FOM-Vorgaben zur Thesis im Online-Campus}
\end{center}
\end{figure}

Laut Herrn Keller sollte der Umfang der Thesis (für eine gute Note) eher im Bereich der 60 Seiten liegen. Wie immer ist das vermutlich mit dem Betreuer abzustimmen. Die Liste der Dozenten, die Abschlussarbeiten betreuen, findet sich auch im \ac{OC}.

Zeit zur Erstellung der Thesis 2-4 Monate.

Es müssen zwei gedruckte Arbeiten abgegeben werden. Flüchtige Quellen als PDF ausgeben lassen und auf CD abgeben. Thesis zusätzlich digital einreichen. Beim Binden der Thesis auf Qualität achten. Haptik und erster Eindruck sind in der Bewertung \enquote{auch} wichtig. Arbeiten können in jedem FOM Studienzentrum abgegeben werden.

\subsection{Post-Abgabephase}
Nach Abgabe ca. 2 Wochen bis zum Kolloquium.

Kolloquium:
\begin{itemize}
\item Dauer: 30 Minuten
\item Präsentation (manche Prüfer wollen eine, andere nicht)
\item Betreuer vorher fragen was er möchte
\item Es gibt einen Frageteil, dieser bezieht sich auf die Arbeit, kann aber auch darüber hinaus gehen.
\item Der Tag des Kolloquiums steht auf der Endbenotung
\item Thesis und Kolloquium sind zwei getrennte Prüfungsbereiche. Für beide gibt es nur zwei Versuche.
\item Am Tag des Kolloquiums erhält man die Bestätigung, ob bestanden oder nicht
\end{itemize}

\newpage
\section{Latex-Details}

\subsection{Verwendete Software, Editor und Zusatzpakete}
\subsubsection{Windows 8+}
\begin{itemize}
\item MikTex: 2.9, 32-bit
\item Biblatex: 3.5, Zusatz: Biber.exe
\item Editor: TexStudio (kann ich empfehlen), Notepad++
\end{itemize}

\subsubsection{Mac OSX und iOS}
\begin{itemize}
\item MacTeX: \url{https://tug.org/mactex}
\item Editor: TexPad \url{https://www.texpadapp.com}
\end{itemize}

\subsubsection{Online}
Overleaf ist eine Online-Anwendung mit der Ihr direkt im Browser an eurer Thesis schreiben könnt. Bis 1GB Größe und maximal 60 Einzeldateien könnt ihr Overleaf kostenlos nutzen: \url{https://www.overleaf.com/}


\subsection{Dokumentenklasse}
Eigentlich hatte Prof. Finke empfohlen die Dokumentklassen \enquote{Book} oder \enquote{Report} für die Erstellung der Bachelor-Thesis zu verwenden, da diese über weitere Gliederungsebenen verfügen. Ich verwende dennoch eine leicht modifizierte Komaskript-Klasse \enquote{scrartcl}, mit der Erweiterung um eine Ebene. Siehe (skripte/weitereEbene.tex). Das Skript stammt irgendwo aus den Netz und übersteigt meine \LaTeX{}-Fähigkeiten. Dadurch kann ich über eine weitere Ebene in der Arbeit verfügen, ohne mich mit der Modifikation von Kapitel-Seiten rumschlagen~\footcite[Vgl. ][Seite 5]{Tanenbaum.2003} zu müssen. Diese Quelle ist nur zur Demonstration und hat keinen inhaltlichen Bezug hierzu. Es werden übrigens nur die Quellen im Literaturverzeichnis angezeigt, die auch referenziert sind.


\subsection{Grafiken}
Das Paket \textbackslash usepackage\{float\} ermöglicht es die Grafiken und Tabellen an der Stelle im Text zu positionieren, wo diese im Quelltext stehen (Option H). Ansonsten würde \LaTeX{} diese dort unterbringen, wo es typographisch sinnvoll wäre - das wollen wir ja nicht ;-).

Die Breite der Grafiken am Besten relativ zum Text angeben.

Da der Pfad zu den Abbildungen im Hauptdokument definiert wurde, muss hier nur noch der Name des Bildes ohne Dateiendung stehen (sup).

% Um Latex die beste Position fuer das Bild bestimmen zu lassen, folgende Parameter verwenden: [htbp]
\begin{figure}[H]
%\begin{center}
\caption{Titel der Abbildung hier}
\includegraphics[width=\textwidth]{sup}
\label{fig:sup}
%\end{center}
\end{figure}


\subsection{Quellcode}
Quellcode kann auf unterschiedliche Arten eingebaut werden (vgl. \autoref{code:imgInclude}).
Zum einen kann es hier durch direktives Einbinden in der Kapitel-Datei geschehen.
\begin{lstlisting}[caption={Bilder und Grafiken einbinden}\label{code:imgInclude}]
% Hier wird aufgezeigt, wie man eine Grafik einbindet, es wird also in der PDF angezeigt,
% da es in einem Quellcode-Listing steht.
% Auch wenn es hier faelschlicherweise als LaTeX-Befehl angezeigt wird.
\includegraphics[width=0.9\textwidth]{sup}
\end{lstlisting}


% JavaScript und TypeScript Showcase:
\begin{minipage}{.49\textwidth}
	\begin{lstlisting}[caption={Hello World in JavaScript}\label{code:JsVsTs},language=JavaScript,numbers=none] 
	   function say(message) {
		  alert(message);	
	   }
	   
	   const words = ['Hello', 'World'];	
	   
	   
	   // say 'Hello World!'
	   say(words.join(' ') + '!'); 
	\end{lstlisting}
	\end{minipage}\hfill
   \begin{minipage}{.49\textwidth}
	\begin{lstlisting}[caption={Hello World in TypeScript}\label{code:TsVsJs},language=TypeScript,numbers=none] 
	   function say(message: string): void {
		  alert(message);	
	   }
	   
	   const words: string[] = 
		   ['Hello', 'World'];
	   
	   // say 'Hello World!'
	   say(words.join(' ') + '!'); 
	\end{lstlisting}
	\end{minipage}

	

\subsection{Tabellen}
\begin{table}[H]
%\centering
\caption{Beispieltabelle 1}
\begin{tabular}[ht]{|l|l|l|}
  \hline
  \textbf{Abkürzung} & \textbf{Beschreibung} & \textbf{Berechnung}\\
  \hline\hline
    MEK & Materialeinzelkosten & \\
  	MGK & Materialgemeinkosten & $+ \uparrow$~*\\
    FEK & Fertigungseinzelkosten & \\
  	FGK & Fertigungsgemeinkosten & $+ \uparrow$~*\\
	SEKF & Sondereinzelkosten der Fertigung & \\
	\hline\hline
	\multicolumn{3}{|l|}{\textbf{= Herstellungskosten}} \\
	\hline\hline
  	VwGK & Verwaltungsgemeinkosten & $+ \uparrow$~*\\
  	VtGK & Vertriebsgemeinkosten & $+ \uparrow$~*\\
  	SEKVt & Sondereinzelkosten des Vertriebes & \\
	\hline\hline
	\multicolumn{3}{|l|}{\textbf{= Selbstkosten}} \\
	\hline\hline
	\multicolumn{3}{|l|}{+ Gewinnaufschlag} \\
	\multicolumn{3}{|l|}{+ Rabatte} \\
	\hline\hline
	\multicolumn{3}{|l|}{\textbf{= Nettoverkaufspreis (NVP)}} \\
	\hline
	\multicolumn{3}{|l|}{+ Umsatzsteuer} \\
	\hline\hline
	\multicolumn{3}{|l|}{\textbf{= Bruttoverkaufspreis (BVP)}} \\
	\hline
\end{tabular} \\
		\label{tbl:beispieltabelle2}
\end{table}

%\clearpage % hiermit werden alle Bilder Tabellen ausgeworfen

Bei längeren Quellcode-Listings empfiehlt es sich jedoch auf eine externe Datei im Ordner Quellcode zu verlinken und diese einzubauen:
\lstinputlisting[language=HTML,caption={HTML als Datei einbinden}]{./Quellcode/Beispiel.html}


\subsection{Todos}
Das Paket todo \todo{schreibt Notizen an die Seite \ldots} ermöglicht es Notizen am Rand der Arbeit zu erzeugen. 

\missingfigure{Fehlende Abbildung.}

Mit dem Befehl missingfigure können Platzhalter Bilder im Text definiert werden.



\subsection{Biblatex}
Von den vielen verfügbaren Literatur-Paketen habe ich mich für Biblatex entschieden. Die Anforderungen der FOM sollten hiermit erfüllt sein. Ich habe bisher nur Einträge \enquote{@book} getestet. Wie immer steckt der Teufel hier im Detail und es wird sich später herausstellen, ob Biblatex eine gute Wahl war. Die Anpassungen hierfür liegen unter skripte/modsBiblatex. Ich verwende das Backend Biber, welches bib-Dateien in UTF-8 verarbeiten kann.

\subsection{Listen und Aufzählungen}
\subsubsection{Listen}
\begin{itemize}
\item ein wichtiger Punkt
\item noch ein wichtiger Punkt
\item und so weiter
\end{itemize}
\subsubsection{Aufzählungen}
\begin{enumerate}
\item Reihenfolge ist hier wichtig
\item Dieser Punkt kommt nach dem ersten
\item Da sollte jetzt eine 3 vorne stehen
\end{enumerate}

\paragraph{Tiefste Ebene 1}
Dies ist die tiefste Gliederungsebene. Sollten doch mehr Ebenen benötigt werden, muss eine andere Dokumentenklasse verwendet werden.

\paragraph{Tiefste Ebene 2}
Der zweite Punkt in dieser Ebene ist zur Erinnerung daran, dass es nie nie niemals nur einen Unterpunkt geben darf.

\subsection{Skript zum Kompilieren}
Latex will ja bekanntlich in einer bestimmten Reihenfolge aufgerufen werden:
\begin{lstlisting}
pdflatex thesis_main.tex
makeindex thesis_main.nlo -s nomencl.ist -o thesis_main.nls
biber thesis_main
pdflatex thesis_main.tex
pdflatex thesis_main.tex
thesis_main.pdf
\end{lstlisting}

Dies ist der Inhalt der Batchdatei \enquote{compile.bat}.

\input{kapitel/fazit/fazit}

%-----------------------------------
% Literaturverzeichnis
%-----------------------------------
\newpage
%\addcontentsline{toc}{section}{Literatur}

\pagenumbering{Roman} %Zähler wieder römisch ausgeben
\setcounter{page}{4}  %Zähler manuell hochsetzen

%\begin{RaggedRight}
\printbibliography
%\end{RaggedRight}

% Alternative Darstellung:
% Literaturverzeichnis nach Typ (@book, @arcticle ...) sortiert.
% Dazu die Zeile (\printbibliography) auskommentieren und folgenden code verwenden:

%\printbibheading
%\printbibliography[type=article,heading=subbibliography,title={Artikel}]
%\printbibliography[type=book,heading=subbibliography,title={Bücher}]
%\printbibliography[type=online,heading=subbibliography,title={Webseiten}]

\newpage
\pagenumbering{gobble} % Keine Seitenzahlen mehr

%-----------------------------------
% Ehrenwörtliche Erklärung
%-----------------------------------
\section*{
	\langde{Ehrenwörtliche Erklärung}
	\langen{Declaration in lieu of oath}}
\langde{Hiermit versichere ich, dass die vorliegende Arbeit von mir selbstständig und ohne unerlaubte Hilfe angefertigt worden ist, insbesondere dass ich alle Stellen, die wörtlich oder annähernd wörtlich aus Veröffentlichungen entnommen sind, durch Zitate als solche gekennzeichnet habe. Ich versichere auch, dass die von mir eingereichte schriftliche Version mit der digitalen Version übereinstimmt. Weiterhin erkläre ich, dass die Arbeit in gleicher oder ähnlicher Form noch keiner Prüfungsbehörde/Prüfungsstelle vorgelegen hat. Ich erkläre mich damit \textcolor{red}{einverstanden/nicht einverstanden}, dass die Arbeit der Öffentlichkeit zugänglich gemacht wird. Ich erkläre mich damit einverstanden, dass die Digitalversion dieser Arbeit zwecks Plagiatsprüfung auf die Server externer Anbieter hochgeladen werden darf. Die Plagiatsprüfung stellt keine Zurverfügungstellung für die Öffentlichkeit dar.}
\langen{I hereby declare that I produced the submitted paper with no assistance from any other party and without the use of any unauthorized aids and, in particular, that I have marked as quotations all passages which are reproduced verbatim or near-verbatim from publications. Also, I declare that the submitted print version of this thesis is identical with its digital version. Further, I declare that this thesis has never been submitted before to any examination board in either its present form or in any other similar version. I herewith agree/disagree that this thesis may be published. I herewith consent that this thesis may be uploaded to the server of external contractors for the purpose of submitting it to the contractors’ plagiarism detection systems. Uploading this thesis for the purpose of submitting it to plagiarism detection systems is not a form of publication.}


\par\medskip
\par\medskip

\vspace{5cm}

\begin{table}[H]
	\centering
	\begin{tabular*}{\textwidth}{c @{\extracolsep{\fill}} ccccc}
		\myOrt, \the\day.\the\month.\the\year
		&
		% Hinterlege deine eingescannte Unterschrift im Verzeichnis /abbildungen und nenne sie unterschrift.png
		% Bilder mit transparentem Hintergrund können teils zu Problemen führen
		\includegraphics[width=0.35\textwidth]{unterschrift}\vspace*{-0.35cm}
		\\
		\rule[0.5ex]{12em}{0.55pt} & \rule[0.5ex]{12em}{0.55pt} \\
		(Ort, Datum) & (Eigenhändige Unterschrift)
		\\
	\end{tabular*} \\
\end{table}

\end{document}
